\documentclass{article} % For LaTeX2e
\usepackage{nips14submit_e,times}
\usepackage{hyperref}
\usepackage{url}
\usepackage{multirow}
%\documentstyle[nips14submit_09,times,art10]{article} % For LaTeX 2.09

\usepackage{framed}
\usepackage{algorithmic}
\usepackage{graphicx} 
\usepackage{amssymb}
\usepackage{amsmath}
\usepackage{amsthm}
\usepackage{caption}
\usepackage{subcaption}
\usepackage{listings}

\lstloadlanguages{Matlab}

\lstset{
  language=Matlab,
  frame=single,
  breaklines=true,
  basicstyle=\tiny,
  postbreak=\raisebox{0ex}[0ex][0ex]{\ensuremath{\color{red}\hookrightarrow\space}}
}

\newcommand{\script}[2]{
\lstinputlisting{#1}
}

\usepackage[compact]{titlesec}
\titlespacing{\section}{0pt}{0.5ex}{0.3ex}
\titlespacing{\subsection}{0pt}{0.2ex}{0ex}
\titlespacing{\subsubsection}{0pt}{0.1ex}{0ex}

\newcommand{\startcompact}[1]{\par\vspace{-0.75em}\begin{#1}%
  \allowdisplaybreaks\ignorespaces}

\newcommand{\stopcompact}[1]{\end{#1}\ignorespaces}

\usepackage{paralist}

\makeatletter
\ifcase \@ptsize \relax% 10pt
  \newcommand{\miniscule}{\@setfontsize\miniscule{4}{5}}% \tiny: 5/6
\or% 11pt
  \newcommand{\miniscule}{\@setfontsize\miniscule{5}{6}}% \tiny: 6/7
\or% 12pt
  \newcommand{\miniscule}{\@setfontsize\miniscule{5}{6}}% \tiny: 6/7
\fi
\makeatother

\newcommand {\aplt} {\ {\raise-.5ex\hbox{$\buildrel<\over\sim$}}\ }

\newcommand{\eqn}[1]{Eqn.~\ref{eqn:#1}}
\newcommand{\fig}[1]{Fig.~\ref{fig:#1}}
\newcommand{\tab}[1]{Table~\ref{tab:#1}}
\newcommand{\secc}[1]{Section~\ref{sec:#1}}
\def\etal{{\textit{et~al.~}}}
\newcommand{\BigO}[1]{\ensuremath{\operatorname{O}\left(#1\right)}}
\usepackage[symbol*]{footmisc}

\DefineFNsymbolsTM{myfnsymbols}{% def. from footmisc.sty "bringhurst" symbols
  \textasteriskcentered *
  \textdagger    \dagger
  \textdaggerdbl \ddagger
  \textsection   \mathsection
  \textbardbl    \|%
  \textparagraph \mathparagraph
}%

\nipsfinalcopy % Uncomment for camera-ready version

\begin{document}

\setcounter{section}{8}

\pagebreak
\section{Supplementary material}
We present the efficient expressions discovered by our system, using
Matlab-style syntax, and we visualize computation trees. Each example contains: (i) code that computes the
original target formulas; (ii) the formulae derived by our system and
(iii) code that verifies the correctness of the expression. The size of
matrices $n$, $m$ can be chosen arbitrary.

Code for generating the expressions can be downloaded from \url{https://github.com/kkurach/math_learning}.
The source files for this paper are available at \url{https://github.com/kkurach/math_learning/paper/}.




\subsection{$\mathbf{(\sum AA^T)_k}$}



{\bf k = 1}

\script{sup/SumAAT_1.m}



{\bf k = 2}

\script{sup/SumAAT_2.m}



{\bf k = 3}

\script{sup/SumAAT_3.m}



{\bf k = 4}

\script{sup/SumAAT_4.m}



{\bf k = 5}

\script{sup/SumAAT_5.m}



{\bf k = 6}

\script{sup/SumAAT_6.m}



{\bf k = 7}

\script{sup/SumAAT_7.m}



{\bf k = 8}

\script{sup/SumAAT_8.m}



{\bf k = 9}

\script{sup/SumAAT_9.m}



{\bf k = 10}

\script{sup/SumAAT_10.m}



{\bf k = 11}

\script{sup/SumAAT_11.m}



{\bf k = 12}

\script{sup/SumAAT_12.m}



{\bf k = 13}

\script{sup/SumAAT_13.m}



{\bf k = 14}

\script{sup/SumAAT_14.m}



{\bf k = 15}

\script{sup/SumAAT_15.m}



\subsection{$\mathbf{(\sum AB)_k}$}



{\bf k = 1}

\script{sup/SumAB_1.m}



{\bf k = 2}

\script{sup/SumAB_2.m}



{\bf k = 3}

\script{sup/SumAB_3.m}



{\bf k = 4}

\script{sup/SumAB_4.m}



{\bf k = 5}

\script{sup/SumAB_5.m}



{\bf k = 6}

\script{sup/SumAB_6.m}



{\bf k = 7}

\script{sup/SumAB_7.m}



{\bf k = 8}

\script{sup/SumAB_8.m}



{\bf k = 9}

\script{sup/SumAB_9.m}



{\bf k = 10}

\script{sup/SumAB_10.m}



{\bf k = 11}

\script{sup/SumAB_11.m}



{\bf k = 12}

\script{sup/SumAB_12.m}



{\bf k = 13}

\script{sup/SumAB_13.m}



{\bf k = 14}

\script{sup/SumAB_14.m}



{\bf k = 15}

\script{sup/SumAB_15.m}



\subsection{$\mathbf{(\sum (A.*A)A^T})_k$}



{\bf k = 1}

\script{sup/SumAmultA_1.m}



{\bf k = 2}

\script{sup/SumAmultA_2.m}



{\bf k = 3}

\script{sup/SumAmultA_3.m}



{\bf k = 4}

\script{sup/SumAmultA_4.m}



{\bf k = 5}

\script{sup/SumAmultA_5.m}



{\bf k = 6}

\script{sup/SumAmultA_6.m}



{\bf k = 7}

\script{sup/SumAmultA_7.m}



{\bf k = 8}

\script{sup/SumAmultA_8.m}



{\bf k = 9}

\script{sup/SumAmultA_9.m}



\subsection{{\bf Sym$_k$}}



{\bf k = 1}

\script{sup/Sym_1.m}



{\bf k = 2}

\script{sup/Sym_2.m}



{\bf k = 3}

\script{sup/Sym_3.m}



{\bf k = 4}

\script{sup/Sym_4.m}



{\bf k = 5}

\script{sup/Sym_5.m}



{\bf k = 6}

\script{sup/Sym_6.m}



{\bf k = 7}

\script{sup/Sym_7.m}



{\bf k = 8}

\script{sup/Sym_8.m}



\subsection{{\bf (RBM-1)$_k$}}



{\bf k = 1}

\script{sup/RBMOneSide_1.m}



{\bf k = 2}

\script{sup/RBMOneSide_2.m}



{\bf k = 3}

\script{sup/RBMOneSide_3.m}



{\bf k = 4}

\script{sup/RBMOneSide_4.m}



{\bf k = 5}

\script{sup/RBMOneSide_5.m}



{\bf k = 6}

\script{sup/RBMOneSide_6.m}



{\bf k = 7}

\script{sup/RBMOneSide_7.m}



{\bf k = 8}

\script{sup/RBMOneSide_8.m}



\subsection{{\bf (RBM-2)$_k$}}



{\bf k = 1}

\script{sup/RBM_1.m}



{\bf k = 2}

\script{sup/RBM_2.m}



{\bf k = 3}

\script{sup/RBM_3.m}



{\bf k = 4}

\script{sup/RBM_4.m}



{\bf k = 5}

\script{sup/RBM_5.m}



{\bf k = 6}

\script{sup/RBM_6.m}



\end{document}


